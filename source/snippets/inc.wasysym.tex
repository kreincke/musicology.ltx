% mycsrf 'for beeing included' snippet template
%
% (c) Karsten Reincke, Frankfurt a.M. 2012, ff.
%
% This text is licensed under the Creative Commons Attribution 3.0 Germany
% License (http://creativecommons.org/licenses/by/3.0/de/): Feel free to share
% (to copy, distribute and transmit) or to remix (to adapt) it, if you respect
% how you must attribute the work in the manner specified by the author(s):
% \newline
% In an internet based reuse please link the reused parts to mycsrf.fodina.de
% and mention the original author Karsten Reincke in a suitable manner. In a
% paper-like reuse please insert a short hint to mycsrf.fodina.de and to the
% original author, Karsten Reincke, into your preface. For normal quotations
% please use the scientific standard to cite
%


%% use all entries of the bibliography

\section{Andere Sonderzeichen: wasysym ($\bigstar\bigstar\bigstar$)}

Gleiches gilt für das \LaTeX-Zusatzpaket
\textit{wasysym}\footnote{\cite[vgl.][\nopage wp]{CtanWasysym2018a}.
Die Paketbeschreibung gibt an, dass \acc{wasysym} unter der \acc{LaTeX\ Project
Public Li­cense} veröffentlicht wird. Das ist eine von der \acc{OSI} anerkannte
Open-Source-Lizenz ($\rightarrow$
\href{https://opensource.org/licenses/LPPL-1.3c}
{https://opensource.org/licenses/LPPL-1.3c}).}: Es erweitert den Vorrat an
musikalischen Zeichen um Notensymbole. Wie in \LaTeX\ üblich, muss das Paket
über ein Kommando in die Präambel eingebunden werden
(\texttt{\textbackslash{usepackage\{wasysym\}}}), damit man danach auf die
Zeichen \eighthnote \ (= \texttt{\small \textbackslash{eighthnote}}),
\quarternote \ (= \texttt{\small \textbackslash{quarternote}}), \halfnote \ (=
\texttt{\small \textbackslash{halfnote}}), \fullnote \ (= \texttt{\small
\textbackslash{fullnote}}), und \twonotes \ (= \texttt{\small
\textbackslash{twonotes}}) zugreifen kann.\footcite[vgl.][2]{Kielhorn2003a}

