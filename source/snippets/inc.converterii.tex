% mycsrf 'for beeing included' snippet template
%
% (c) Karsten Reincke, Frankfurt a.M. 2012, ff.
%
% This text is licensed under the Creative Commons Attribution 3.0 Germany
% License (http://creativecommons.org/licenses/by/3.0/de/): Feel free to share
% (to copy, distribute and transmit) or to remix (to adapt) it, if you respect
% how you must attribute the work in the manner specified by the author(s):
% \newline
% In an internet based reuse please link the reused parts to mycsrf.fodina.de
% and mention the original author Karsten Reincke in a suitable manner. In a
% paper-like reuse please insert a short hint to mycsrf.fodina.de and to the
% original author, Karsten Reincke, into your preface. For normal quotations
% please use the scientific standard to cite
%

\section{Konverter II}

Vor der großen Verknüpfung vom Backend und Frontend müssen wir nun noch 
eruieren, welche der üblicherweise genannt Konverter wir wirklich woher beziehen können:

\subsection{abc2ly}

\acc{abc2ly} wird von \acc{Lilypond} im Paket als Tool mitgeliefert und kann
nach dessen Installation über die Kommandozeile aufgerufen werden.\footnote{\ra\
\href{http://lilypond.org/doc/v2.18/Documentation/usage/invoking-abc2ly}
{http://lilypond.org/doc/v2.18/Documentation/usage/invoking-abc2ly}}

\subsection{abc2mtex}

\acc{abc2mtex} ist 'nur' noch als CTAN-Paket
erhältlich\footcite[vgl.][\nopage wp.]{abc2mtex2019a}. Der auf der Paketseite
integrierte Link auf die eigentliche Homepage ist nicht mehr aktuell, anders der
Link auf die Quellen und der Link auf das Handbuch: Folgt man ersterem, erhält
man Zugriff auf das \acc{tar.gz-Quellpaket}; letzterer macht den \acc{Userguide}
zugänglich. Ein Binärpaket gibt es allerdings nicht (mehr). Das downloadbare
Quellenpaket enthält jedoch ein Makefile, über das das eigentliche Program in
Verbindung mit \acc{gcc} und \acc{make} umstandslos erzeugt werden kann.


\subsection{ly2abc}

Der Konverter \acc{ly2abc} wird als Github-Projekt entwickelt und distribuiert.
Auf der entsprechenden Homepage wird allerdings gesagt, es werde daran noch
stark gearbeitet und es sei kein Support für Mehrstimmigkeit
geplant.\footcite[vgl.][\nopage wp.]{ly2abc2019a} Damit darf dieser Konverter
getrost ignoriert werden: Zur Verwendung von Harmonieanalysen in und mit \LaTeX\
kann er -- unabhängig von seiner sonstigen Relevanz -- nichts beitragen.

\subsection{musicxml2ly}

\acc{musicxml2ly} wird von \acc{Lilypond} im Paket als Tool mitgeliefert und kann
nach dessen Installation über die Kommandozeile aufgerufen werden.\footnote{\ra\
\href{http://lilypond.org/doc/v2.18/Documentation/usage/invoking-musicxml2ly}
{http://lilypond.org/doc/v2.18/Documentation/usage/invoking-musicxml2ly}}

\subsection{mxml2abc}

Sucht man nach \acc{mxml2abc}, führt das Netz einen zunächst in Sackgassen. Es
sagt, der Konverter \textit{mxml2abc} sei als unabhängiges Projekt angelegt; das
Tool solle ein Java-Programm sein.\footnote{\ra\
\href{http://freshmeat.sourceforge.net/projects/mxml2abc}
{http://freshmeat.sourceforge.net/projects/mxml2abc} RDL 2019-04-11} Und es
zeigt noch alte Updateankündigungen an.\footnote{\ra\
\href{https://musescore.org/en/node/15698} {https://musescore.org/en/node/15698}
RDL 2019-04-11} Allerdings ist die angegebene Homepage von \acc{mxml2abc},
selbst nicht mehr erreichbar.\footnote{\ra\
\href{https://code.google.com/p/mxml2abc/}
{https://code.google.com/p/mxml2abc/} RDL 2019-04-11}
  
Schließlich führt es den hartnäckigen Sucher aber auch zu dem Konverter
\acc{mxml2abc} von Stephen Merrony\footnote{\ra\
\href{https://www.softpaz.com/software/download-mxml2abc-windows-42792.htm}
{https://www.softpaz.com/software/download-mxml2abc-windows-42792.htm}}. Das
dort angebotene Java-Programm kann per \texttt{java -jar
mxml2abc-stephen-merrony.jar musicxml.xml} aufgerufen werden und schreibt den
konvertierten Inhalt der Datei \acc{musicxml.xml} auf die Konsole zurück.

\subsection{xml2abc}

Der Konverter \acc{xml2ab} wird mit \acc{EasyABC} mitgeliefert. Zu starten ist
er im Sourcecode-Ordner per \texttt{python xml2abc musicxml.xml}, nimmt dann die
auf der Kommandozeile übergegebe MusicXML-Datei und konvertiert deren Inhalt 
in \acc{abc}-Code.

\subsection{xml2ly}

Der Konverter \acc{xml2ly} wird gesondert bereitgestellt, und zwar in Form einer
\acc{xsl}-Datei.\footcite[vgl.][\nopage wp.]{xml2ly2019a} Um die als Konverter zu
nutzen, muss man einen zusätzlichen xslt-Prozessor installiert haben.
Unglücklicherweise verweist der Link auf der Homepage, der in den
Downloadbereich führen soll, in einen leeren Raum. Auch der Link, der das
ausgewiesene Downloadpaket direkt zugänglich zugänglich machen soll,
funktioniert nicht. Hier scheint einiges veraltet zu sein. Das passt zum
Erscheinungsjahr (2002) und zur Releasenummer (0.0.34) der 'letzten'
Veröffentlichung.\footnote{Allerdings scheint der xml2ly-Code -- der Idee von
Opensource-Software entsprechend -- später in einer Bibliothek weiterverndet
worden sein. \ra\
\href{https://github.com/grame-cncm/libmusicxml/releases}{https://github.com/grame-cncm/libmusicxml/releases
}}

Damit kann dieser Konverter nicht ausprobiert und also nicht wirklich als Option
in Rechnung gestellt werden. 

\subsection{xml2pmx}

Der Konverter \acc{xml2pmx} wird als Paket über das Icking-Music-Archive
bereitgestellt\footcite[vgl.][\nopage wp.]{xml2pmx2019a}, das auch auf seiner
Softwareseite selbst auf den Konverter verweist.\footnote{\ra\
\href{https://icking-music-archive.org/software/htdocs/index.html}
{https://icking-music-archive.org/software/htdocs/index.html}} Dieser kann als
Windows- oder Linuxpaket heruntergeladen werden. Es enthält dann die
entsprechende Binärversion und eine Aufrufskript, das ein Beispiel konvertiert.
Die eigenen Umformungen können dann entsprechend umgesetzt werden.

% this is only inserted to eject fault messages in texlipse
%\bibliography{../bib/literature}
